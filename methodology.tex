\section{Methodology}
The general approach taken to prune an optimally trained neural network in the present work is to create a ranked list of all the neurons in the network based off of one of the 3 ranking criteria we discuss further in this section. This ranking is done using a validation dataset which is different from the dataset used for training the network. The effects of removing an increasing percentage of neurons based off their ranks are then analysed in the results section.

\subsection{Brute Force Removal Approach}
This is perhaps the most naive yet the most accurate method for pruning the network. It is also the slowest and hence unusable on large-scale neural networks with thousands of neurons. The idea is to manually check the effect of every single neuron on the output. This is done by running a forward propagation on the validation set $K$ times (where $K$ is the total number of neurons in the network), turning off exactly one neuron each time (keeping all other neurons active) and noting down the change in error. Turning a neuron off can be achieved by simply setting its output to 0. This results in all the outgoing weights from that neuron being turned off. This change in error is then used to generate the ranked list. 

\subsubsection{Taylor Series Representation of Error}
Let us denote the total error from the optimally trained neural network for any given validation dataset with $N$ instances as $\Etotal$. Then,
\begin{align}
\Etotal = \sum _n E_n,
\end{align}
where $E_n$ is the error from the network over one validation instance. $E_n$ can be seen as a function $O$, where $O$ is the output of any general neuron in the network (In reality this error depends on each neuron's output, but for the sake of simplicity we use $O$ to represent that). This error can be approximated at a particular neuron's output (say $O_k$) by using the 2nd order Taylor Series as,

\begin{align}
\hat E_n(O) \approx E_n(O_k) + (O-O_k)\cdot E_n(O_k) + \left.\pdv{E_n}{O}\right|_{O_k} +  0.5\cdot (O-O_k)^2\cdot \left.\pdv[2]{E_n}{O}\right|_{O_k}\label{eq:ts1},
\end{align}

where $\hat E_n(O_k)$ represents the contribution of a neuron $k$ to the total error $E_n$ of the network for any given validation instance $n$. When this neuron is pruned, its output $O_k$ becomes 0. From equation \ref{eq:ts1}, the contribution $E_n(0)$ of this neuron, then becomes:

\begin{align}
\hat E_n(0) \approx E_n(O_k) - O_k\cdot \left.\pdv{E_n}{O}\right|_{O_k} +  0.5\cdot O_k^2\cdot \left.\pdv[2]{E_n}{O}\right|_{O_k}\label{eq:ts2}
\end{align}

Replacing $O$ by $O_k$ in equation \ref{eq:ts1} shows us that the error is approximated perfectly by equation \ref{eq:ts1} at $O_k$. Using this and equation \ref{eq:ts2} we get:

\begin{align}
\Delta E_{n,k} = \hat E_n(0) - \hat E_n(O_k)= - O_k\cdot \left.\pdv{E_n}{O}\right|_{O_k} + 0.5\cdot O_k^2\cdot \left.\pdv[2]{E_n}{O}\right|_{O_k}\label{eq:ts3},
\end{align}

where $\Delta E_{n,k}$ is the change in the total error of the network given a validation instance $n$, when exactly one neuron ($k$) is turned off.


\subsection{Linear Approximation Approach}
\begin{figure}[bh!]
\centering
\newcommand{\repSigmoid}{$\sigma(\cdot)$}
\newcommand{\repLinear}{$\sum$}
\newcommand{\repMse}{MSE}
\newcommand{\repFirstSum}{$\Input j 1$}
\newcommand{\repLastSum}{$\Input i 0$}
\newcommand{\repFirstOutput}{\hspace{1.5cm}$\Con j i 0 \!=\! \Weight j i 0 \Out j 1$}
\newcommand{\repLastOutput}{$\Out i 0$}
\newcommand{\repLoss}{$E$}
\def\svgwidth{0.9\textwidth}
\hspace{-2cm}
\import{}{drawing.pdf_tex}
\hspace{-2cm}
\caption{Simple feed-forward network illustrating the naming of different variables, where $\sigma(\cdot)$ is the nonlinearity, MSE is the mean-squared error cost function and $E$ is the overall loss.}
\end{figure}
We define the following network terminology here which will be used in this section and all subsequent sections  unless stated otherwise. Figure \ref{fig:comp_graph} can be used as a reference to the terminology defined here:

\begin{align}
E &= \frac{1}{2}\sum\limits_i (\Out i 0 - \Target i)^2 &
\Out i m &= \sigma(\Input i m) &
\Input i m &= \sum\limits_j {\Weight j i m}{ \Out j {m + 1}} &
\Con j i m = \Weight j i m \Out j {m+1}\label{eq:term}
\end{align}
Superscripts represent the index of the layer of the network in question, with 0 representing the output layer. $E$ is the squared-error network cost function. Note that we are dropping the $E_n$ notation used previously as the subsequent discussion is insusceptible to the data instances. $\Out i m$ is the $i$th output in layer $m$ generated by the activation function $\sigma$, which in this paper is is the standard logistic sigmoid. $\Input i m$ is the weighted sum of inputs to the $i$th neuron in the $m$th layer, and $\Con j i m$ is the contribution of the $j$th neuron in the $(m+1)$th layer to the input of the $i$th neuron in the $m$th layer. $\Weight j i m$ is the weight between the $j$th neuron in the $(m+1)$th layer and the $i$th neuron in the $m$th layer.

We can use equation \ref{eq:ts3} to get the linear error approximation of the change in error due to the $k$th neuron being turned off and represent it as $\Delta E_{k}^1$ as follows:

\begin{align}
\Delta E_{k}^1 = - o_k\cdot \left.\pdv{E}{{\Out j {m+1}}}\right|_{o_k}
\end{align}

The derivative term above is the first-order gradient which represents the change in error with respect to the output of a given neuron $o_j$ in the $(m+1)$th layer. This term can be collected during back-propagation. The derivative term above can be calculated as follows:

\begin{align}
\pdv{E}{{\Out j {m+1}}} = \sum\limits_i \pdv{E}{\Input i m}\cdot \Weight j i m
\end{align}

The full step-by-step mathematical derivation of the above equation can be found in the appendix.
\subsection{Quadratic Approximation Approach}

As seen in equation \ref{eq:ts3}, $\Delta E_{n,k}$ which can now be represented as $\Delta E_{k}^2$ is the quadratic approximation of the change in error due to the $k$th neuron being turned off. The quadratic term in equation \ref{eq:ts3} requires some discussion which we provide here. A more detailed and step-by-step mathematical derivation can be found in the appendix.

Let us reproduce equation \ref{eq:ts3} in our new terminology here: 
\begin{align}
\Delta E_{k}^2 = - o_k\cdot \left.\pdv{E}{{\Out j {m+1}}}\right|_{o_k} + 0.5\cdot o_k^2\cdot \left.\pdv[2]{E}{{\Out j {m+1}}}\right|_{o_k}\
\end{align}

The second term here involves the second-order gradient which represents the second-order change in error with respect to the output of a given neuron $o_j$ in the $(m+1)$th layer. This term can be generated by performing back-propagation using second derivatives. A full derivation of the second derivative back-propagation can be found in the appendix. We will quote some results from the derivation here. The second-order derivative term can be represented as:

\begin{align}
\pdv[2]{E}{{\Out j {m+1}}} &= \sum_i
\pdv[2]{E}{{\Con j i m}} \left({\Weight j i m}\right)^2
\end{align} 

Here,$\Con j i m$ is one of the component terms of $\Input i m$, as follows from the equations in \ref{eq:term}. Hence, it can be easily proved that (full proof in appendix):
\begin{align}
\pdv[2]{E}{{\Con j i m}} = \pdv[2]{E}{{\Input i m}}
\end{align}

Now, the value of $\Input i m$ can be easily calculated through the steps of the second-order back-propagation using Chain Rule. The full derivation can again, be found in the appendix.
\begin{align}
\pdv[2]{E}{{\Input i m}}=\pdv[2]{E}{{\Out i m}} \left(\sigma^{\prime}\left({\Input i m}\right)\right)^2
+
\pdv{E}{{\Out i m}}\sigma^{\prime\prime}\left(\Input i m\right)
\end{align}

\subsection{Pruning Algorithm: The Gain-switch}
We propose the Gain-switch algorithm for pruning an optimally trained neural network here. We define gain as the quadratic error approximation $\Delta E_{k}^2$ we got from the Taylor Series.

The first step is to  decide a stopping criterion. This can vary depending on the application but some intuitive stopping criteria can be the maximum number of neurons to remove, percentage scaling needed, maximum allowable accuracy drop etc. The Gain-switch algorithm performs the pruning in a greedy manner. It performs a forward propagation followed by a second-order back-propagation and collects the linear and quadratic gradients. It is to be noted that there is no weight update taking place during the back-propagation step as the network is already trained.This step is only used to collect the gradients. The algorithm then ranks all the neurons in the network based on their respective gain values and removes the neuron with the least value of the gain. 

\begin{algorithm}[H]
 \KwData{optimally trained network, training set}
 \KwResult{A pruned network after applying the Gain-switch algorithm}
 initialize and define stopping criterion \;
 \While{stopping criterion is not met}{
 $gain$ = $\Delta E_{k}^2$ \;
  perform forward propagation over the training set \;
  perform second-order back-propagation without updating weights and collect linear and quadratic gradients \;
  rank the remaining neurons based on $gain$ \;
  remove the neuron with the least value of $gain$ \;
 }
 \caption{The proposed Gain-switch pruning algorithm}
\end{algorithm}

The advantage of taking a greedy approach is that while removing the neurons, we are take into account the dependencies the neurons might have with one another. 
